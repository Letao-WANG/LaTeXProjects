\documentclass[12pt]{fphw}

% Template-specific packages
\usepackage[utf8]{inputenc} % Required for inputting international characters
\usepackage[T1]{fontenc} % Output font encoding for international characters
\usepackage{mathpazo} % Use the Palatino font
\usepackage{graphicx} % Required for including images
\usepackage{booktabs} % Required for better horizontal rules in tables
\usepackage{listings} % Required for insertion of code
\usepackage{enumerate} % To modify the enumerate environment

%%%%%%%%%%%%%%%%%%%%%%%%%%%%%%%%%%

\usepackage[english]{babel}
\usepackage{amsmath}
\usepackage{graphicx}
\usepackage{float}
\usepackage[colorinlistoftodos]{todonotes}

\usepackage{xcolor}

\setlength{\parindent}{0em}

\begin{document}
\title{Travaux dirigés } % Assignment title
\author{Letao WANG}
\begin{titlepage}

\newcommand{\HRule}{\rule{\linewidth}{0.5mm}} % Defines a new command for the horizontal lines, change thickness here

\center % Center everything on the page
 
%----------------------------------------------------------------------------------------
%	HEADING SECTIONS
%----------------------------------------------------------------------------------------

\textsc{\large UFR Mathématiques et Informatique }\\[1.5cm] % Name of your university/college
\includegraphics[scale=.2]{logo_u-paris_tex_regular.png}\\[1cm] % Include a department/university logo - this will require the graphicx package
\textsc{\Large Algorithmique avancée}\\[0.5cm] % Major heading such as course name
\textsc{\large IF05X040}\\[0.5cm] % Minor heading such as course title

%----------------------------------------------------------------------------------------
%	TITLE SECTION
%----------------------------------------------------------------------------------------

\HRule \\[0.4cm]
{ \huge \bfseries Travaux Dirigés 6-7}\\[0.4cm] % Title of your document
\HRule \\[1.5cm]
 
%----------------------------------------------------------------------------------------
%	AUTHOR SECTION
%----------------------------------------------------------------------------------------

\begin{minipage}{0.4\textwidth}
\begin{flushleft} \large
\emph{Prof: }Nicolas \textsc{Loménie }\\
\emph{Author: }Letao \textsc{Wang}\\
\end{flushleft}

\end{minipage}\\[2cm]

% If you don't want a supervisor, uncomment the two lines below and remove the section above
%\Large \emph{Author:}\\
%John \textsc{Smith}\\[3cm] % Your name

%----------------------------------------------------------------------------------------
%	DATE SECTION
%----------------------------------------------------------------------------------------

{\large \today}\\[2cm] % Date, change the \today to a set date if you want to be precise

\vfill % Fill the rest of the page with whitespace

\end{titlepage}

%%%%%%%%%%%%%%%%%%%%%%%%%%%%%%%%%%%%%%%%

%----------------------------------------------------------------------------------------
%	ASSIGNMENT CONTENT
%----------------------------------------------------------------------------------------

\part*{Partie exercices}
\section*{Exercice 2.7.}
\begin{problem}
Déterminer un arbre couvrant de poids minimal pour le graphe de la
figure 2.3
\end{problem}
\subsection*{Réponse}

On peut utiliser l'algo de Prim et Kruskal, les resultat sont meme,
le poids minimal est 12.

\section*{Exercice 2.8.}
\begin{problem}
On considère un graphe valué G et un ensemble U de sommets. Proposer
un algorithme qui détermine un arbre couvrant de poids minimal parmi ceux tels que tous les sommets de U sont des feuilles.\\
Démontrer sa validité et préciser sa complexité.
\end{problem}
\subsection*{Réponse}

\section*{Exercice 2.9.}
\begin{center}
	\includegraphics[width=1\columnwidth]{Figure2.4.png} % Example image
\end{center}
\subsection*{Réponse}
\begin{center}
l’algorithme de Dijkstra\\
	\begin{tabular}{l l l l l l l l l l}
		\toprule
		\textit{Sommet} & u & a & b & c & d &e & f & g & v\\
		\midrule
		Distance &  \textcolor{red}{0} & 1 &3 & $\infty$ & $\infty$  & $\infty$  & $\infty$  & $\infty$  & $\infty$  \\
		Distance &  \textcolor{red}{0} & \textcolor{red}{1} &3 & 4 & 3  & $\infty$  & $\infty$  & $\infty$  & $\infty$  \\
		Distance &  \textcolor{red}{0} & \textcolor{red}{1} &\textcolor{red}{3} & 4 & \textcolor{red}{3}  & 4  & 6  & 7  & $\infty$  \\
		Distance &  \textcolor{red}{0} & \textcolor{red}{1} &\textcolor{red}{3} & \textcolor{red}{4} & \textcolor{red}{3}  & \textcolor{red}{4}  & 6  & 6  & $\infty$  \\
		Distance &  \textcolor{red}{0} & \textcolor{red}{1} &\textcolor{red}{3} & \textcolor{red}{4} & \textcolor{red}{3}  & \textcolor{red}{4}  & \textcolor{red}{6}  & \textcolor{red}{6}  & 7  \\
		Distance &  \textcolor{red}{0} & \textcolor{red}{1} &\textcolor{red}{3} & \textcolor{red}{4} & \textcolor{red}{3}  & \textcolor{red}{4}  & \textcolor{red}{6}  & \textcolor{red}{6}  & \textcolor{red}{7}  \\
		\bottomrule
	\end{tabular}
\end{center}

\section*{Exercice 2.10}
\begin{problem}
A partir de l’algorithme de Dijkstra, construire un algorithme qui
détermine le diamètre de chaque composante connexe d’un graphe. Quelle en est la
complexité ?
\end{problem}
\subsection*{Réponse}

On fait l'algo Dijkstra pour tous les sommets  dans chaque composante connexe, retourner le plus grand nombre.

Complexity : n * algo Dijkstra ( on considere c'est $O(m + n log n)$  à l’aide de tas de Fibonacci) \\

Donc c'est $ O(n(m+n log n))$

\section*{Exercice 3.1.}
\subsection*{Réponse}
\begin{center}
	\begin{tabular}{l l l l}
		\toprule
		\textit{} & G1 & G2 & G3 \\
		\midrule
		Chemin eulérien & true & true & false\\
		Cycle eulérien & false & true & false\\
		Chemin hamiltonien & true & false & true\\
		Cycle hamiltonien & true & false & false\\
		\bottomrule
	\end{tabular}
\end{center}

\section*{Exercice 3.2.}

\begin{center}
	\includegraphics[width=0.5\columnwidth]{Figure3.2.png} % Example image
\end{center}

\subsection*{Réponse}Au premier, il faut trouver les sommets qui degre est impair. Donc on a sommet F, B, G, D. Ensuite on construire le Figure 3.2a\\

\begin{center}
	\includegraphics[width=0.5\columnwidth]{Figure3.2a.png} % Example image
\end{center}

Selon le Figure 3.2a, on veut connecter les 4 sommets avec le minimal poid, on peut choisir Edge(F, G) et Edge(B, D).\\

On peut ajouter un Edge(B, D) dans le Figure 3.2, donc il y a seulement 2 sommets impaire, on peut trouver un chemin eulerien.\\

Le poids minimal total est l'addition de tous les poids de Figure3.2 et Edge(F,G), Edge(B,D).\\

Tous les poids de Figure3.2 : \(1+4+2+1+1+3+2+2+5+1+3 = 25\)\\
Avec les 2 autre edges : \(25 + 3 + 4 = 32\)\\

Donc le poids minimal total est \textbf{32}.

\newpage
\part*{Partie codes}

\section*{TP6.B. Visualisation de Graphes}
\subsection*{JGraphT}
Je utilise le bibliotheque JGraphT pour visualiser graphe. Une partie de mon code d'affichage.

\lstinputlisting[
		caption=GrapheGenerate, % Caption above the listing
		label=lst:luftballons, % Label for referencing this listing
		language=Java, % Use Perl functions/syntax highlighting
		frame=single, % Frame around the code listing
		showstringspaces=false, % Don't put marks in string spaces
		numbers=left, % Line numbers on left
		numberstyle=\tiny, % Line numbers styling
	]{GraphGenerate.java}

\newpage

Voici le resultat du code:\\
\begin{center}
	\includegraphics[width=0.6\columnwidth]{graph.png} % Example image
\end{center}

\subsection*{Neo4j}
Neo4j est une base de données non relationnelle, une base de données graphique native. Dans une base de données relationnelle normale, la recherche d'un élément de données nécessite souvent l'interrogation de plusieurs tables de la base de données, en particulier pour les grands projets qui nécessitent de nombreuses et longues requêtes sautantes. Les performances de ce type de base de données sont réduites dans ce scénario. La base de données de graphes introduit le concept de Edge entre les Vertex, ce qui améliore considérablement les performances dans ce scénario.\\

\textbf{Neo4j Java}
\lstinputlisting[
		caption=Neo4j avec Java exemple, % Caption above the listing
		label=lst:luftballons, % Label for referencing this listing
		language=Java, % Use Perl functions/syntax highlighting
		frame=single, % Frame around the code listing
		showstringspaces=false, % Don't put marks in string spaces
		numbers=left, % Line numbers on left
		numberstyle=\tiny, % Line numbers styling
	]{NeoJava.java}
Voici le resultat dans le terminale:\\
Node1 relation12 Node2




\end{document}
