\documentclass{fphw}
\usepackage{hyperref} 
\setlength{\parindent}{0em}
\begin{document}
\title{Questions}
\part*{Certaines questions des definitions d'algo}
\section{Les definitions de AlgoAvanceeCoursPart1.pdf}
\subsection*{Page 15}
Un \textbf{chemin} de longueur $k$ entre deux sommets $u$ et $v$ est une suite de $k$ aretes ($u_i, u_{i+1} $) tels que $u_0$ = $u$, $u_k$ = $v$ et tous les $u_i$ sont disjoints.\\
Dans le cas de graphes orientes, un \textbf{chemin oriente} necessite l’orientation des
aretes dans le sens  $\overrightarrow{u_i u_{i+1}}$.
Si les aretes et les sommets ne sont pas tous disjoints, on parle de \textbf{marche} entre
$u$ et $v$.

\subsection*{Page 17}
Un cycle de longueur $k$ est une suite de $k$ arêtes $\left(u_{i}, u_{i+1}\right)$ tels que $u_{0}=u_{k}$ et tous les $u_{i}$ sont disjoints.
Dans le cas de graphes orientés, un \textbf{cycle orienté} nécessite l'orientation des arêtes dans le sens $\overrightarrow{u_{i} u_{i+1}}$
	
	
\section{Les definitions que je trouve dans Wikipedia}
\begin{center}
\href{https://fr.wikipedia.org/wiki/Chemin_(th%C3%A9orie_des_graphes)}{Lien de Chemin (théorie des graphes) \textbf{Wikipedia}}\\


\end{center}

Dans un graphe orienté, un \textbf{chemin} d'origine $x$ et d'extrémité $y$, noté $\mu[x, y]$, est défini par une suite finie d'arcs consécutifs, reliant $x$ à $y .$ La notion correspondante dans les graphes non orientés est celle de \textbf{ chaîne}.\\
Un \textbf{ chemin élémentaire} est un chemin ne passant pas deux fois par un même sommet, c'est-à-dire dont tous les sommets sont distincts.
\\
\begin{center}
\href{https://fr.wikipedia.org/wiki/Cycle_(th%C3%A9orie_des_graphes))}{Lien de Cycle (théorie des graphes) \textbf{Wikipedia}}
\end{center}
Dans un graphe non orienté, un \textbf{cycle} est une suite d'arêtes consécutives (chaine simple) dont les deux sommets extrémités sont identiques. Dans les graphes orientés, la notion équivalente est celle de \textbf{circuit}, même si on parle parfois aussi de cycle (par exemple dans l'expression graphe acyclique orienté).
\section{Pourquoi la définition du cours est complètement différente de celle de Wikipedia ? }
Il est clair que de nombreux definitions de la théorie des graphes sont définis différemment dans le cours et dans Wikipédia.\\

Si je comprends bien, le cours définit Chemin avec sommet disjoint, mais Wikipedia ne dit pas ca, La définition du cours de Chemin est plus proche de la définition Wikipedia du Chemin élémentaire.\\

La même situation se produit également dans le cycle, sur lequel je vous ai demande après la dernière cours a lundi, mais votre déclaration est différente de celle de Wikipedia. Je sais que certains des concepts ont peut-être un sens très large, mais ces définitions confuses affectent sérieusement ma compréhension de la théorie des graphes et je voudrais vous demande pourriez-vous expliquer \textbf{pourquoi la définition du cours est complètement différente de celle de Wikipedia ?}\\

\textbf{Merci merci beaucoup d'avance.}

Cordialement,\\
Letao WANG

\end{document}


